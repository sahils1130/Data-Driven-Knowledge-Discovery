% Options for packages loaded elsewhere
\PassOptionsToPackage{unicode}{hyperref}
\PassOptionsToPackage{hyphens}{url}
%
\documentclass[
]{article}
\usepackage{lmodern}
\usepackage{amssymb,amsmath}
\usepackage{ifxetex,ifluatex}
\ifnum 0\ifxetex 1\fi\ifluatex 1\fi=0 % if pdftex
  \usepackage[T1]{fontenc}
  \usepackage[utf8]{inputenc}
  \usepackage{textcomp} % provide euro and other symbols
\else % if luatex or xetex
  \usepackage{unicode-math}
  \defaultfontfeatures{Scale=MatchLowercase}
  \defaultfontfeatures[\rmfamily]{Ligatures=TeX,Scale=1}
\fi
% Use upquote if available, for straight quotes in verbatim environments
\IfFileExists{upquote.sty}{\usepackage{upquote}}{}
\IfFileExists{microtype.sty}{% use microtype if available
  \usepackage[]{microtype}
  \UseMicrotypeSet[protrusion]{basicmath} % disable protrusion for tt fonts
}{}
\makeatletter
\@ifundefined{KOMAClassName}{% if non-KOMA class
  \IfFileExists{parskip.sty}{%
    \usepackage{parskip}
  }{% else
    \setlength{\parindent}{0pt}
    \setlength{\parskip}{6pt plus 2pt minus 1pt}}
}{% if KOMA class
  \KOMAoptions{parskip=half}}
\makeatother
\usepackage{xcolor}
\IfFileExists{xurl.sty}{\usepackage{xurl}}{} % add URL line breaks if available
\IfFileExists{bookmark.sty}{\usepackage{bookmark}}{\usepackage{hyperref}}
\hypersetup{
  pdftitle={Analysis of Online Shoppers' Purchase Intention},
  pdfauthor={Kaarthik Sundaramoorthy, Sahil Shah and Vidhi Shah},
  hidelinks,
  pdfcreator={LaTeX via pandoc}}
\urlstyle{same} % disable monospaced font for URLs
\usepackage[margin=1in]{geometry}
\usepackage{color}
\usepackage{fancyvrb}
\newcommand{\VerbBar}{|}
\newcommand{\VERB}{\Verb[commandchars=\\\{\}]}
\DefineVerbatimEnvironment{Highlighting}{Verbatim}{commandchars=\\\{\}}
% Add ',fontsize=\small' for more characters per line
\usepackage{framed}
\definecolor{shadecolor}{RGB}{248,248,248}
\newenvironment{Shaded}{\begin{snugshade}}{\end{snugshade}}
\newcommand{\AlertTok}[1]{\textcolor[rgb]{0.94,0.16,0.16}{#1}}
\newcommand{\AnnotationTok}[1]{\textcolor[rgb]{0.56,0.35,0.01}{\textbf{\textit{#1}}}}
\newcommand{\AttributeTok}[1]{\textcolor[rgb]{0.77,0.63,0.00}{#1}}
\newcommand{\BaseNTok}[1]{\textcolor[rgb]{0.00,0.00,0.81}{#1}}
\newcommand{\BuiltInTok}[1]{#1}
\newcommand{\CharTok}[1]{\textcolor[rgb]{0.31,0.60,0.02}{#1}}
\newcommand{\CommentTok}[1]{\textcolor[rgb]{0.56,0.35,0.01}{\textit{#1}}}
\newcommand{\CommentVarTok}[1]{\textcolor[rgb]{0.56,0.35,0.01}{\textbf{\textit{#1}}}}
\newcommand{\ConstantTok}[1]{\textcolor[rgb]{0.00,0.00,0.00}{#1}}
\newcommand{\ControlFlowTok}[1]{\textcolor[rgb]{0.13,0.29,0.53}{\textbf{#1}}}
\newcommand{\DataTypeTok}[1]{\textcolor[rgb]{0.13,0.29,0.53}{#1}}
\newcommand{\DecValTok}[1]{\textcolor[rgb]{0.00,0.00,0.81}{#1}}
\newcommand{\DocumentationTok}[1]{\textcolor[rgb]{0.56,0.35,0.01}{\textbf{\textit{#1}}}}
\newcommand{\ErrorTok}[1]{\textcolor[rgb]{0.64,0.00,0.00}{\textbf{#1}}}
\newcommand{\ExtensionTok}[1]{#1}
\newcommand{\FloatTok}[1]{\textcolor[rgb]{0.00,0.00,0.81}{#1}}
\newcommand{\FunctionTok}[1]{\textcolor[rgb]{0.00,0.00,0.00}{#1}}
\newcommand{\ImportTok}[1]{#1}
\newcommand{\InformationTok}[1]{\textcolor[rgb]{0.56,0.35,0.01}{\textbf{\textit{#1}}}}
\newcommand{\KeywordTok}[1]{\textcolor[rgb]{0.13,0.29,0.53}{\textbf{#1}}}
\newcommand{\NormalTok}[1]{#1}
\newcommand{\OperatorTok}[1]{\textcolor[rgb]{0.81,0.36,0.00}{\textbf{#1}}}
\newcommand{\OtherTok}[1]{\textcolor[rgb]{0.56,0.35,0.01}{#1}}
\newcommand{\PreprocessorTok}[1]{\textcolor[rgb]{0.56,0.35,0.01}{\textit{#1}}}
\newcommand{\RegionMarkerTok}[1]{#1}
\newcommand{\SpecialCharTok}[1]{\textcolor[rgb]{0.00,0.00,0.00}{#1}}
\newcommand{\SpecialStringTok}[1]{\textcolor[rgb]{0.31,0.60,0.02}{#1}}
\newcommand{\StringTok}[1]{\textcolor[rgb]{0.31,0.60,0.02}{#1}}
\newcommand{\VariableTok}[1]{\textcolor[rgb]{0.00,0.00,0.00}{#1}}
\newcommand{\VerbatimStringTok}[1]{\textcolor[rgb]{0.31,0.60,0.02}{#1}}
\newcommand{\WarningTok}[1]{\textcolor[rgb]{0.56,0.35,0.01}{\textbf{\textit{#1}}}}
\usepackage{graphicx,grffile}
\makeatletter
\def\maxwidth{\ifdim\Gin@nat@width>\linewidth\linewidth\else\Gin@nat@width\fi}
\def\maxheight{\ifdim\Gin@nat@height>\textheight\textheight\else\Gin@nat@height\fi}
\makeatother
% Scale images if necessary, so that they will not overflow the page
% margins by default, and it is still possible to overwrite the defaults
% using explicit options in \includegraphics[width, height, ...]{}
\setkeys{Gin}{width=\maxwidth,height=\maxheight,keepaspectratio}
% Set default figure placement to htbp
\makeatletter
\def\fps@figure{htbp}
\makeatother
\setlength{\emergencystretch}{3em} % prevent overfull lines
\providecommand{\tightlist}{%
  \setlength{\itemsep}{0pt}\setlength{\parskip}{0pt}}
\setcounter{secnumdepth}{-\maxdimen} % remove section numbering
\usepackage{booktabs}
\usepackage{longtable}
\usepackage{array}
\usepackage{multirow}
\usepackage{wrapfig}
\usepackage{float}
\usepackage{colortbl}
\usepackage{pdflscape}
\usepackage{tabu}
\usepackage{threeparttable}
\usepackage{threeparttablex}
\usepackage[normalem]{ulem}
\usepackage{makecell}
\usepackage{xcolor}

\title{Analysis of Online Shoppers' Purchase Intention}
\author{Kaarthik Sundaramoorthy, Sahil Shah and Vidhi Shah}
\date{6/23/2020}

\begin{document}
\maketitle

\hypertarget{milestone-1-project-proposal}{%
\subsection{\texorpdfstring{\center Milestone 1: Project proposal
\center }{Milestone 1: Project proposal }}\label{milestone-1-project-proposal}}

\hypertarget{problem-description}{%
\subsubsection{Problem Description}\label{problem-description}}

In the era of online shopping, known as e-shopping, people use online
transactions to buy the items they need while exploring it online. This
helps the buyers as well as sellers to understand the patterns,
intentions, and behavior of various online customers. Thereby, helping
businesses improve their revenue by focusing on customer experiences and
marketing. Hence, the analysis of online shoppers' purchase intention
has become an emerging field in data mining. Click-stream analysis
refers to the online shoppers' behavior analysis as they invoke a
sequence of web pages in a particular session. Therefore, analyzing this
data is a primary goal for successful online businesses as they extract
the clicks and behavior through web page requests. Our proposed solution
is to provide a decisive and feasible recommendation algorithm that will
allow us to predict the behavior of the shoppers'.

The dataset used in the project is based on \emph{``online shoppers
purchasing intention''} available on UCI Machine Learning dataset.\\

URL :
\url{https://archive.ics.uci.edu/ml/datasets/Online+Shoppers+Purchasing+Intention+Dataset}

\hypertarget{importing-libraries}{%
\paragraph{\texorpdfstring{Importing Libraries\\
}{Importing Libraries }}\label{importing-libraries}}

This are the important libraries that are to be installed for the
execution of the file.

\begin{Shaded}
\begin{Highlighting}[]
\KeywordTok{library}\NormalTok{(ggplot2)}
\KeywordTok{library}\NormalTok{(tidyverse)}
\KeywordTok{library}\NormalTok{(gmodels)}
\KeywordTok{library}\NormalTok{(dplyr)}
\KeywordTok{library}\NormalTok{(ggmosaic)}
\KeywordTok{library}\NormalTok{(corrplot)}
\KeywordTok{library}\NormalTok{(caret)}
\KeywordTok{library}\NormalTok{(rpart)}
\KeywordTok{library}\NormalTok{(rpart.plot)}
\KeywordTok{library}\NormalTok{(cluster)}
\KeywordTok{library}\NormalTok{(fpc)}
\KeywordTok{library}\NormalTok{(data.table)}
\KeywordTok{library}\NormalTok{(knitr)}
\KeywordTok{library}\NormalTok{(kableExtra)}
\end{Highlighting}
\end{Shaded}

\hypertarget{importing-the-dataset}{%
\paragraph{\texorpdfstring{Importing the Dataset\\
}{Importing the Dataset }}\label{importing-the-dataset}}

The \texttt{read.csv()} command is used to import the dataset.

\begin{Shaded}
\begin{Highlighting}[]
\NormalTok{dataset <-}\StringTok{ }\KeywordTok{read.csv}\NormalTok{(}\StringTok{"online_shoppers_intention.csv"}\NormalTok{, }\DataTypeTok{header =} \OtherTok{TRUE}\NormalTok{)}
\KeywordTok{attach}\NormalTok{(dataset)}
\end{Highlighting}
\end{Shaded}

Checking the number of columns and rows of the dataset.

\begin{Shaded}
\begin{Highlighting}[]
\KeywordTok{ncol}\NormalTok{(dataset)}
\end{Highlighting}
\end{Shaded}

\begin{verbatim}
## [1] 18
\end{verbatim}

\begin{Shaded}
\begin{Highlighting}[]
\KeywordTok{nrow}\NormalTok{(dataset)}
\end{Highlighting}
\end{Shaded}

\begin{verbatim}
## [1] 12330
\end{verbatim}

Looking at the dataset data structure.

\begin{Shaded}
\begin{Highlighting}[]
\KeywordTok{str}\NormalTok{(dataset)}
\end{Highlighting}
\end{Shaded}

\begin{verbatim}
## 'data.frame':    12330 obs. of  18 variables:
##  $ Administrative         : int  0 0 0 0 0 0 0 1 0 0 ...
##  $ Administrative_Duration: num  0 0 0 0 0 0 0 0 0 0 ...
##  $ Informational          : int  0 0 0 0 0 0 0 0 0 0 ...
##  $ Informational_Duration : num  0 0 0 0 0 0 0 0 0 0 ...
##  $ ProductRelated         : int  1 2 1 2 10 19 1 0 2 3 ...
##  $ ProductRelated_Duration: num  0 64 0 2.67 627.5 ...
##  $ BounceRates            : num  0.2 0 0.2 0.05 0.02 ...
##  $ ExitRates              : num  0.2 0.1 0.2 0.14 0.05 ...
##  $ PageValues             : num  0 0 0 0 0 0 0 0 0 0 ...
##  $ SpecialDay             : num  0 0 0 0 0 0 0.4 0 0.8 0.4 ...
##  $ Month                  : Factor w/ 10 levels "Aug","Dec","Feb",..: 3 3 3 3 3 3 3 3 3 3 ...
##  $ OperatingSystems       : int  1 2 4 3 3 2 2 1 2 2 ...
##  $ Browser                : int  1 2 1 2 3 2 4 2 2 4 ...
##  $ Region                 : int  1 1 9 2 1 1 3 1 2 1 ...
##  $ TrafficType            : int  1 2 3 4 4 3 3 5 3 2 ...
##  $ VisitorType            : Factor w/ 3 levels "New_Visitor",..: 3 3 3 3 3 3 3 3 3 3 ...
##  $ Weekend                : logi  FALSE FALSE FALSE FALSE TRUE FALSE ...
##  $ Revenue                : logi  FALSE FALSE FALSE FALSE FALSE FALSE ...
\end{verbatim}

\begin{Shaded}
\begin{Highlighting}[]
\KeywordTok{summary}\NormalTok{(dataset)}
\end{Highlighting}
\end{Shaded}

\begin{verbatim}
##  Administrative   Administrative_Duration Informational    
##  Min.   : 0.000   Min.   :   0.00         Min.   : 0.0000  
##  1st Qu.: 0.000   1st Qu.:   0.00         1st Qu.: 0.0000  
##  Median : 1.000   Median :   7.50         Median : 0.0000  
##  Mean   : 2.315   Mean   :  80.82         Mean   : 0.5036  
##  3rd Qu.: 4.000   3rd Qu.:  93.26         3rd Qu.: 0.0000  
##  Max.   :27.000   Max.   :3398.75         Max.   :24.0000  
##                                                            
##  Informational_Duration ProductRelated   ProductRelated_Duration
##  Min.   :   0.00        Min.   :  0.00   Min.   :    0.0        
##  1st Qu.:   0.00        1st Qu.:  7.00   1st Qu.:  184.1        
##  Median :   0.00        Median : 18.00   Median :  598.9        
##  Mean   :  34.47        Mean   : 31.73   Mean   : 1194.8        
##  3rd Qu.:   0.00        3rd Qu.: 38.00   3rd Qu.: 1464.2        
##  Max.   :2549.38        Max.   :705.00   Max.   :63973.5        
##                                                                 
##   BounceRates         ExitRates         PageValues        SpecialDay     
##  Min.   :0.000000   Min.   :0.00000   Min.   :  0.000   Min.   :0.00000  
##  1st Qu.:0.000000   1st Qu.:0.01429   1st Qu.:  0.000   1st Qu.:0.00000  
##  Median :0.003112   Median :0.02516   Median :  0.000   Median :0.00000  
##  Mean   :0.022191   Mean   :0.04307   Mean   :  5.889   Mean   :0.06143  
##  3rd Qu.:0.016813   3rd Qu.:0.05000   3rd Qu.:  0.000   3rd Qu.:0.00000  
##  Max.   :0.200000   Max.   :0.20000   Max.   :361.764   Max.   :1.00000  
##                                                                          
##      Month      OperatingSystems    Browser           Region     
##  May    :3364   Min.   :1.000    Min.   : 1.000   Min.   :1.000  
##  Nov    :2998   1st Qu.:2.000    1st Qu.: 2.000   1st Qu.:1.000  
##  Mar    :1907   Median :2.000    Median : 2.000   Median :3.000  
##  Dec    :1727   Mean   :2.124    Mean   : 2.357   Mean   :3.147  
##  Oct    : 549   3rd Qu.:3.000    3rd Qu.: 2.000   3rd Qu.:4.000  
##  Sep    : 448   Max.   :8.000    Max.   :13.000   Max.   :9.000  
##  (Other):1337                                                    
##   TrafficType               VisitorType     Weekend         Revenue       
##  Min.   : 1.00   New_Visitor      : 1694   Mode :logical   Mode :logical  
##  1st Qu.: 2.00   Other            :   85   FALSE:9462      FALSE:10422    
##  Median : 2.00   Returning_Visitor:10551   TRUE :2868      TRUE :1908     
##  Mean   : 4.07                                                            
##  3rd Qu.: 4.00                                                            
##  Max.   :20.00                                                            
## 
\end{verbatim}

The purchasing intention model is designed as a classification problem
which measures the purchasers' commitment to finalize purchase intent.
Hence we have the session data of the users which has two categories :
users who purchased the item and who didn't. The dataset consists of
both numerical data and categorical data, and thus the target value is
categorical. Table 1 refers to the numerical features and Table 2 refers
to the categorical features used in the prediction model respectively.
There are a total of 12,330 rows where each row represents session data
of one particular user.

\begin{Shaded}
\begin{Highlighting}[]
\NormalTok{tab1 <-}\StringTok{ }\KeywordTok{read.csv}\NormalTok{(}\StringTok{"table1.csv"}\NormalTok{, }\DataTypeTok{header =} \OtherTok{TRUE}\NormalTok{)}
\KeywordTok{kable}\NormalTok{(tab1) }\OperatorTok
\StringTok{  }\KeywordTok{kable_styling}\NormalTok{(}\DataTypeTok{full_width =}\NormalTok{ T)}
\end{Highlighting}
\end{Shaded}

\begin{tabu} to \linewidth {>{\raggedright}X>{\raggedright}X>{\raggedleft}X>{\raggedleft}X>{\raggedleft}X}
\hline
ï..Feature.Name & Description & Min..value & Max..value & SD\\
\hline
Administrative & Number of pages visited by the visitor about account management & 0 & 27.0 & 3.322e+00\\
\hline
Administrative duration & Total amount of time (in seconds) spent by the visitor on account management related pages & 0 & 3399.0 & 1.768e+02\\
\hline
Informational & Number of pages visited by the visitor about Web site, communication and address information of the shopping site & 0 & 24.0 & 1.270e+00\\
\hline
Informational duration & Total amount of time (in seconds) spent by the visitor on informational pages & 0 & 2549.4 & 1.407e+02\\
\hline
Product related & Number of pages visited by visitor about product related pages & 0 & 705.0 & 4.448e+01\\
\hline
Product related duration & Total amount of time (in seconds) spent by the visitor on product related pages & 0 & 63974.0 & 1.914e+03\\
\hline
Bounce rates & Average bounce rate value of the pages visited by the visitor & 0 & 0.2 & 4.849e-02\\
\hline
Exit rate & Average exit rate value of the pages visited by the visitor & 0 & 0.2 & 4.860e-02\\
\hline
Page value & Average page value of the pages visited by the visitor & 0 & 361.8 & 1.857e+01\\
\hline
Special day & Closeness of the site visiting time to a special day & 0 & 1.0 & 1.989e-01\\
\hline
\end{tabu}

\begin{Shaded}
\begin{Highlighting}[]
\NormalTok{tab2 <-}\StringTok{ }\KeywordTok{read.csv}\NormalTok{(}\StringTok{"table2.csv"}\NormalTok{, }\DataTypeTok{header =} \OtherTok{TRUE}\NormalTok{)}
\KeywordTok{kable}\NormalTok{(tab2) }\OperatorTok
\StringTok{  }\KeywordTok{kable_styling}\NormalTok{(}\DataTypeTok{full_width =}\NormalTok{ T)}
\end{Highlighting}
\end{Shaded}

\begin{tabu} to \linewidth {>{\raggedright}X>{\raggedright}X>{\raggedleft}X}
\hline
ï..Name & Description & Values\\
\hline
OperatingSystems & Operating system of the visitor & 8\\
\hline
Browser & Browser of the visitor & 13\\
\hline
Region & Geographic region from which the session has been started by the visitor & 9\\
\hline
TrafficType & Traffic source by which the visitor has arrived at the Web site (e.g., banner, SMS, direct) & 20\\
\hline
VisitorType & Visitor type as New Visitor, Returning Visitor, and Other & 3\\
\hline
Weekend & Boolean value indicating whether the date of the visit is weekend & 2\\
\hline
Month & Month value of the visit date & 10\\
\hline
Revenue & Class label indicating whether the visit has been finalized with a transaction & 2\\
\hline
\end{tabu}

Taking the look at the \textbf{REVENUE} column which is the target
column. The datatype of the REVENUE column is Logical which holds the
value \textbf{TRUE} and \textbf{FALSE}.

\begin{Shaded}
\begin{Highlighting}[]
\KeywordTok{library}\NormalTok{(gmodels)}
\KeywordTok{summary}\NormalTok{(dataset}\OperatorTok{$}\NormalTok{Revenue)}
\end{Highlighting}
\end{Shaded}

\begin{verbatim}
##    Mode   FALSE    TRUE 
## logical   10422    1908
\end{verbatim}

\begin{Shaded}
\begin{Highlighting}[]
\KeywordTok{CrossTable}\NormalTok{(dataset}\OperatorTok{$}\NormalTok{Revenue)}
\end{Highlighting}
\end{Shaded}

\begin{verbatim}
## 
##  
##    Cell Contents
## |-------------------------|
## |                       N |
## |         N / Table Total |
## |-------------------------|
## 
##  
## Total Observations in Table:  12330 
## 
##  
##           |     FALSE |      TRUE | 
##           |-----------|-----------|
##           |     10422 |      1908 | 
##           |     0.845 |     0.155 | 
##           |-----------|-----------|
## 
## 
## 
## 
\end{verbatim}

Adding the new \emph{Revenue\_binary} column by using Logical Data of
Shopper's Revenue into binary dependent variable that will helpful for
potential regression models. The data will be converted with values 0
and 1, i.e.~If it is false the value is 0 and if true it will be 1.

\begin{Shaded}
\begin{Highlighting}[]
\NormalTok{dataset <-}\StringTok{ }\NormalTok{dataset }\OperatorTok
\StringTok{  }\KeywordTok{mutate}\NormalTok{(}\DataTypeTok{Revenue_binary =} \KeywordTok{ifelse}\NormalTok{(dataset}\OperatorTok{$}\NormalTok{Revenue }\OperatorTok{==}\StringTok{ "TRUE"}\NormalTok{, }\DecValTok{1}\NormalTok{, }\DecValTok{0}\NormalTok{))}
\end{Highlighting}
\end{Shaded}

Checking the dataset if it has any missing values.

\begin{Shaded}
\begin{Highlighting}[]
\KeywordTok{colSums}\NormalTok{(}\KeywordTok{is.na}\NormalTok{(dataset))}
\end{Highlighting}
\end{Shaded}

\begin{verbatim}
##          Administrative Administrative_Duration           Informational 
##                       0                       0                       0 
##  Informational_Duration          ProductRelated ProductRelated_Duration 
##                       0                       0                       0 
##             BounceRates               ExitRates              PageValues 
##                       0                       0                       0 
##              SpecialDay                   Month        OperatingSystems 
##                       0                       0                       0 
##                 Browser                  Region             TrafficType 
##                       0                       0                       0 
##             VisitorType                 Weekend                 Revenue 
##                       0                       0                       0 
##          Revenue_binary 
##                       0
\end{verbatim}

\hypertarget{visualizations}{%
\paragraph{\texorpdfstring{Visualizations\\
}{Visualizations }}\label{visualizations}}

\hypertarget{month}{%
\subparagraph{Month}\label{month}}

\begin{Shaded}
\begin{Highlighting}[]
\NormalTok{dataset }\OperatorTok
\StringTok{  }\KeywordTok{ggplot}\NormalTok{() }\OperatorTok{+}\StringTok{ }
\StringTok{  }\KeywordTok{aes}\NormalTok{(}\DataTypeTok{x =}\NormalTok{ Month, }\DataTypeTok{Revenue =}\NormalTok{ ..count..}\OperatorTok{/}\KeywordTok{nrow}\NormalTok{(dataset), }\DataTypeTok{fill =}\NormalTok{ Revenue) }\OperatorTok{+}
\StringTok{  }\KeywordTok{geom_bar}\NormalTok{() }\OperatorTok{+}
\StringTok{  }\KeywordTok{ylab}\NormalTok{(}\StringTok{"Frequency"}\NormalTok{)}
\end{Highlighting}
\end{Shaded}

\includegraphics{m2_files/figure-latex/vis-1.pdf}

The plot describes the frequency of the revenue generated over the
months.

\begin{Shaded}
\begin{Highlighting}[]
\NormalTok{table_month =}\StringTok{ }\KeywordTok{table}\NormalTok{(dataset}\OperatorTok{$}\NormalTok{Month, dataset}\OperatorTok{$}\NormalTok{Revenue)}
\NormalTok{tab_mon =}\StringTok{  }\KeywordTok{as.data.frame}\NormalTok{(}\KeywordTok{prop.table}\NormalTok{(table_month,}\DecValTok{2}\NormalTok{))}
\KeywordTok{colnames}\NormalTok{(tab_mon) =}\StringTok{ }\KeywordTok{c}\NormalTok{(}\StringTok{"Month"}\NormalTok{, }\StringTok{"Revenue"}\NormalTok{, }\StringTok{"perc"}\NormalTok{)}
\KeywordTok{ggplot}\NormalTok{(}\DataTypeTok{data =}\NormalTok{ tab_mon, }\KeywordTok{aes}\NormalTok{(}\DataTypeTok{x =}\NormalTok{ Month, }\DataTypeTok{y =}\NormalTok{ perc, }\DataTypeTok{fill =}\NormalTok{ Revenue)) }\OperatorTok{+}\StringTok{ }
\StringTok{  }\KeywordTok{geom_bar}\NormalTok{(}\DataTypeTok{stat =} \StringTok{'identity'}\NormalTok{, }\DataTypeTok{position =} \StringTok{'dodge'}\NormalTok{, }\DataTypeTok{alpha =} \DecValTok{2}\OperatorTok{/}\DecValTok{3}\NormalTok{) }\OperatorTok{+}\StringTok{ }
\StringTok{  }\KeywordTok{xlab}\NormalTok{(}\StringTok{"Month"}\NormalTok{)}\OperatorTok{+}
\StringTok{  }\KeywordTok{ylab}\NormalTok{(}\StringTok{"Percent"}\NormalTok{)}
\end{Highlighting}
\end{Shaded}

\includegraphics{m2_files/figure-latex/vis2-1.pdf}

The plot portrays the high shopping rates in the months September,
October and November with respect to the customers not buying the
products. These months are comparatively considered as the \emph{Holiday
Season Months}. Also, there is high hits on the website with positive
revenue in the month of may.

\hypertarget{visitor}{%
\subparagraph{Visitor}\label{visitor}}

\begin{Shaded}
\begin{Highlighting}[]
\KeywordTok{theme_set}\NormalTok{(}\KeywordTok{theme_bw}\NormalTok{())}

\CommentTok{## setting default parameters for mosaic plots}
\NormalTok{mosaic_theme =}\StringTok{ }\KeywordTok{theme}\NormalTok{(}\DataTypeTok{axis.text.x =} \KeywordTok{element_text}\NormalTok{(}\DataTypeTok{angle =} \DecValTok{90}\NormalTok{,}
                                                \DataTypeTok{hjust =} \DecValTok{1}\NormalTok{,}
                                                \DataTypeTok{vjust =} \FloatTok{0.5}\NormalTok{),}
                     \DataTypeTok{axis.text.y =} \KeywordTok{element_blank}\NormalTok{(),}
                     \DataTypeTok{axis.ticks.y =} \KeywordTok{element_blank}\NormalTok{())}
\end{Highlighting}
\end{Shaded}

\begin{Shaded}
\begin{Highlighting}[]
\NormalTok{dataset }\OperatorTok\StringTok{ }
\StringTok{  }\KeywordTok{ggplot}\NormalTok{() }\OperatorTok{+}
\StringTok{  }\KeywordTok{geom_mosaic}\NormalTok{(}\KeywordTok{aes}\NormalTok{(}\DataTypeTok{x =} \KeywordTok{product}\NormalTok{(Revenue, VisitorType), }\DataTypeTok{fill =}\NormalTok{ Revenue)) }\OperatorTok{+}
\StringTok{  }\NormalTok{mosaic_theme }\OperatorTok{+}
\StringTok{  }\KeywordTok{xlab}\NormalTok{(}\StringTok{"Visitor Type"}\NormalTok{) }\OperatorTok{+}
\StringTok{  }\KeywordTok{ylab}\NormalTok{(}\OtherTok{NULL}\NormalTok{)}
\end{Highlighting}
\end{Shaded}

\includegraphics{m2_files/figure-latex/vis3-1.pdf}

The comparison of the VisitorType which are New\_Visitors,
Returning\_Visitor and Others with Revenue generated. There are many
returning visitors in the contrast to less new visitors. Although, the
new visitors have high probablity of purchasing the product and help the
revenue than the returning visitors.

\hypertarget{weekend}{%
\subparagraph{Weekend}\label{weekend}}

\begin{Shaded}
\begin{Highlighting}[]
\KeywordTok{CrossTable}\NormalTok{(dataset}\OperatorTok{$}\NormalTok{Weekend, dataset}\OperatorTok{$}\NormalTok{Revenue)}
\end{Highlighting}
\end{Shaded}

\begin{verbatim}
## 
##  
##    Cell Contents
## |-------------------------|
## |                       N |
## | Chi-square contribution |
## |           N / Row Total |
## |           N / Col Total |
## |         N / Table Total |
## |-------------------------|
## 
##  
## Total Observations in Table:  12330 
## 
##  
##                 | dataset$Revenue 
## dataset$Weekend |     FALSE |      TRUE | Row Total | 
## ----------------|-----------|-----------|-----------|
##           FALSE |      8053 |      1409 |      9462 | 
##                 |     0.381 |     2.080 |           | 
##                 |     0.851 |     0.149 |     0.767 | 
##                 |     0.773 |     0.738 |           | 
##                 |     0.653 |     0.114 |           | 
## ----------------|-----------|-----------|-----------|
##            TRUE |      2369 |       499 |      2868 | 
##                 |     1.257 |     6.864 |           | 
##                 |     0.826 |     0.174 |     0.233 | 
##                 |     0.227 |     0.262 |           | 
##                 |     0.192 |     0.040 |           | 
## ----------------|-----------|-----------|-----------|
##    Column Total |     10422 |      1908 |     12330 | 
##                 |     0.845 |     0.155 |           | 
## ----------------|-----------|-----------|-----------|
## 
## 
\end{verbatim}

\begin{Shaded}
\begin{Highlighting}[]
\NormalTok{dataset }\OperatorTok
\StringTok{  }\KeywordTok{ggplot}\NormalTok{() }\OperatorTok{+}
\StringTok{  }\NormalTok{mosaic_theme }\OperatorTok{+}
\StringTok{  }\KeywordTok{geom_mosaic}\NormalTok{(}\KeywordTok{aes}\NormalTok{(}\DataTypeTok{x =} \KeywordTok{product}\NormalTok{(Revenue,Weekend), }\DataTypeTok{fill =}\NormalTok{ Revenue)) }\OperatorTok{+}
\StringTok{  }\KeywordTok{xlab}\NormalTok{(}\StringTok{"Weekend"}\NormalTok{) }\OperatorTok{+}
\StringTok{  }\KeywordTok{ylab}\NormalTok{(}\OtherTok{NULL}\NormalTok{)}
\end{Highlighting}
\end{Shaded}

\includegraphics{m2_files/figure-latex/weekend-1.pdf}

The \textbf{Weekend} analysis shows that more than 70\% of visitors are
visiting the site on weekdays, with 15\% chance of actually buying the
products. The rest 30\% visit on the weekend and there is 17\%
speculation of buying.

\hypertarget{appendixcode}{%
\section{Appendix---Code}\label{appendixcode}}

\begin{Shaded}
\begin{Highlighting}[]
\NormalTok{knitr}\OperatorTok{::}\NormalTok{opts_chunk}\OperatorTok{$}\KeywordTok{set}\NormalTok{(}\DataTypeTok{echo=} \OtherTok{TRUE}\NormalTok{, }\DataTypeTok{warning=}\OtherTok{FALSE}\NormalTok{, }\DataTypeTok{message=}\OtherTok{FALSE}\NormalTok{)}
\KeywordTok{library}\NormalTok{(ggplot2)}
\KeywordTok{library}\NormalTok{(tidyverse)}
\KeywordTok{library}\NormalTok{(gmodels)}
\KeywordTok{library}\NormalTok{(dplyr)}
\KeywordTok{library}\NormalTok{(ggmosaic)}
\KeywordTok{library}\NormalTok{(corrplot)}
\KeywordTok{library}\NormalTok{(caret)}
\KeywordTok{library}\NormalTok{(rpart)}
\KeywordTok{library}\NormalTok{(rpart.plot)}
\KeywordTok{library}\NormalTok{(cluster)}
\KeywordTok{library}\NormalTok{(fpc)}
\KeywordTok{library}\NormalTok{(data.table)}
\KeywordTok{library}\NormalTok{(knitr)}
\KeywordTok{library}\NormalTok{(kableExtra)}
\NormalTok{dataset <-}\StringTok{ }\KeywordTok{read.csv}\NormalTok{(}\StringTok{"online_shoppers_intention.csv"}\NormalTok{, }\DataTypeTok{header =} \OtherTok{TRUE}\NormalTok{)}
\KeywordTok{attach}\NormalTok{(dataset)}
\KeywordTok{ncol}\NormalTok{(dataset)}
\KeywordTok{nrow}\NormalTok{(dataset)}
\KeywordTok{str}\NormalTok{(dataset)}
\KeywordTok{summary}\NormalTok{(dataset)}
\NormalTok{tab1 <-}\StringTok{ }\KeywordTok{read.csv}\NormalTok{(}\StringTok{"table1.csv"}\NormalTok{, }\DataTypeTok{header =} \OtherTok{TRUE}\NormalTok{)}
\KeywordTok{kable}\NormalTok{(tab1) }\OperatorTok
\StringTok{  }\KeywordTok{kable_styling}\NormalTok{(}\DataTypeTok{full_width =}\NormalTok{ T)}
\NormalTok{tab2 <-}\StringTok{ }\KeywordTok{read.csv}\NormalTok{(}\StringTok{"table2.csv"}\NormalTok{, }\DataTypeTok{header =} \OtherTok{TRUE}\NormalTok{)}
\KeywordTok{kable}\NormalTok{(tab2) }\OperatorTok
\StringTok{  }\KeywordTok{kable_styling}\NormalTok{(}\DataTypeTok{full_width =}\NormalTok{ T)}
\KeywordTok{library}\NormalTok{(gmodels)}
\KeywordTok{summary}\NormalTok{(dataset}\OperatorTok{$}\NormalTok{Revenue)}
\KeywordTok{CrossTable}\NormalTok{(dataset}\OperatorTok{$}\NormalTok{Revenue)}
\NormalTok{dataset <-}\StringTok{ }\NormalTok{dataset }\OperatorTok
\StringTok{  }\KeywordTok{mutate}\NormalTok{(}\DataTypeTok{Revenue_binary =} \KeywordTok{ifelse}\NormalTok{(dataset}\OperatorTok{$}\NormalTok{Revenue }\OperatorTok{==}\StringTok{ "TRUE"}\NormalTok{, }\DecValTok{1}\NormalTok{, }\DecValTok{0}\NormalTok{))}
\KeywordTok{colSums}\NormalTok{(}\KeywordTok{is.na}\NormalTok{(dataset))}
\NormalTok{dataset }\OperatorTok
\StringTok{  }\KeywordTok{ggplot}\NormalTok{() }\OperatorTok{+}\StringTok{ }
\StringTok{  }\KeywordTok{aes}\NormalTok{(}\DataTypeTok{x =}\NormalTok{ Month, }\DataTypeTok{Revenue =}\NormalTok{ ..count..}\OperatorTok{/}\KeywordTok{nrow}\NormalTok{(dataset), }\DataTypeTok{fill =}\NormalTok{ Revenue) }\OperatorTok{+}
\StringTok{  }\KeywordTok{geom_bar}\NormalTok{() }\OperatorTok{+}
\StringTok{  }\KeywordTok{ylab}\NormalTok{(}\StringTok{"Frequency"}\NormalTok{)}
\NormalTok{table_month =}\StringTok{ }\KeywordTok{table}\NormalTok{(dataset}\OperatorTok{$}\NormalTok{Month, dataset}\OperatorTok{$}\NormalTok{Revenue)}
\NormalTok{tab_mon =}\StringTok{  }\KeywordTok{as.data.frame}\NormalTok{(}\KeywordTok{prop.table}\NormalTok{(table_month,}\DecValTok{2}\NormalTok{))}
\KeywordTok{colnames}\NormalTok{(tab_mon) =}\StringTok{ }\KeywordTok{c}\NormalTok{(}\StringTok{"Month"}\NormalTok{, }\StringTok{"Revenue"}\NormalTok{, }\StringTok{"perc"}\NormalTok{)}
\KeywordTok{ggplot}\NormalTok{(}\DataTypeTok{data =}\NormalTok{ tab_mon, }\KeywordTok{aes}\NormalTok{(}\DataTypeTok{x =}\NormalTok{ Month, }\DataTypeTok{y =}\NormalTok{ perc, }\DataTypeTok{fill =}\NormalTok{ Revenue)) }\OperatorTok{+}\StringTok{ }
\StringTok{  }\KeywordTok{geom_bar}\NormalTok{(}\DataTypeTok{stat =} \StringTok{'identity'}\NormalTok{, }\DataTypeTok{position =} \StringTok{'dodge'}\NormalTok{, }\DataTypeTok{alpha =} \DecValTok{2}\OperatorTok{/}\DecValTok{3}\NormalTok{) }\OperatorTok{+}\StringTok{ }
\StringTok{  }\KeywordTok{xlab}\NormalTok{(}\StringTok{"Month"}\NormalTok{)}\OperatorTok{+}
\StringTok{  }\KeywordTok{ylab}\NormalTok{(}\StringTok{"Percent"}\NormalTok{)}
\KeywordTok{theme_set}\NormalTok{(}\KeywordTok{theme_bw}\NormalTok{())}

\CommentTok{## setting default parameters for mosaic plots}
\NormalTok{mosaic_theme =}\StringTok{ }\KeywordTok{theme}\NormalTok{(}\DataTypeTok{axis.text.x =} \KeywordTok{element_text}\NormalTok{(}\DataTypeTok{angle =} \DecValTok{90}\NormalTok{,}
                                                \DataTypeTok{hjust =} \DecValTok{1}\NormalTok{,}
                                                \DataTypeTok{vjust =} \FloatTok{0.5}\NormalTok{),}
                     \DataTypeTok{axis.text.y =} \KeywordTok{element_blank}\NormalTok{(),}
                     \DataTypeTok{axis.ticks.y =} \KeywordTok{element_blank}\NormalTok{())}
\NormalTok{dataset }\OperatorTok\StringTok{ }
\StringTok{  }\KeywordTok{ggplot}\NormalTok{() }\OperatorTok{+}
\StringTok{  }\KeywordTok{geom_mosaic}\NormalTok{(}\KeywordTok{aes}\NormalTok{(}\DataTypeTok{x =} \KeywordTok{product}\NormalTok{(Revenue, VisitorType), }\DataTypeTok{fill =}\NormalTok{ Revenue)) }\OperatorTok{+}
\StringTok{  }\NormalTok{mosaic_theme }\OperatorTok{+}
\StringTok{  }\KeywordTok{xlab}\NormalTok{(}\StringTok{"Visitor Type"}\NormalTok{) }\OperatorTok{+}
\StringTok{  }\KeywordTok{ylab}\NormalTok{(}\OtherTok{NULL}\NormalTok{)}
\KeywordTok{CrossTable}\NormalTok{(dataset}\OperatorTok{$}\NormalTok{Weekend, dataset}\OperatorTok{$}\NormalTok{Revenue)}
\NormalTok{dataset }\OperatorTok
\StringTok{  }\KeywordTok{ggplot}\NormalTok{() }\OperatorTok{+}
\StringTok{  }\NormalTok{mosaic_theme }\OperatorTok{+}
\StringTok{  }\KeywordTok{geom_mosaic}\NormalTok{(}\KeywordTok{aes}\NormalTok{(}\DataTypeTok{x =} \KeywordTok{product}\NormalTok{(Revenue,Weekend), }\DataTypeTok{fill =}\NormalTok{ Revenue)) }\OperatorTok{+}
\StringTok{  }\KeywordTok{xlab}\NormalTok{(}\StringTok{"Weekend"}\NormalTok{) }\OperatorTok{+}
\StringTok{  }\KeywordTok{ylab}\NormalTok{(}\OtherTok{NULL}\NormalTok{)}
\end{Highlighting}
\end{Shaded}

\end{document}
